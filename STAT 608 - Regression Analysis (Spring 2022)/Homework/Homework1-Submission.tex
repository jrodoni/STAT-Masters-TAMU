% Options for packages loaded elsewhere
\PassOptionsToPackage{unicode}{hyperref}
\PassOptionsToPackage{hyphens}{url}
%
\documentclass[
]{article}
\usepackage{amsmath,amssymb}
\usepackage{lmodern}
\usepackage{ifxetex,ifluatex}
\ifnum 0\ifxetex 1\fi\ifluatex 1\fi=0 % if pdftex
  \usepackage[T1]{fontenc}
  \usepackage[utf8]{inputenc}
  \usepackage{textcomp} % provide euro and other symbols
\else % if luatex or xetex
  \usepackage{unicode-math}
  \defaultfontfeatures{Scale=MatchLowercase}
  \defaultfontfeatures[\rmfamily]{Ligatures=TeX,Scale=1}
\fi
% Use upquote if available, for straight quotes in verbatim environments
\IfFileExists{upquote.sty}{\usepackage{upquote}}{}
\IfFileExists{microtype.sty}{% use microtype if available
  \usepackage[]{microtype}
  \UseMicrotypeSet[protrusion]{basicmath} % disable protrusion for tt fonts
}{}
\makeatletter
\@ifundefined{KOMAClassName}{% if non-KOMA class
  \IfFileExists{parskip.sty}{%
    \usepackage{parskip}
  }{% else
    \setlength{\parindent}{0pt}
    \setlength{\parskip}{6pt plus 2pt minus 1pt}}
}{% if KOMA class
  \KOMAoptions{parskip=half}}
\makeatother
\usepackage{xcolor}
\IfFileExists{xurl.sty}{\usepackage{xurl}}{} % add URL line breaks if available
\IfFileExists{bookmark.sty}{\usepackage{bookmark}}{\usepackage{hyperref}}
\hypersetup{
  pdftitle={STAT 608 - Assignment1},
  pdfauthor={Jack Rodoni},
  hidelinks,
  pdfcreator={LaTeX via pandoc}}
\urlstyle{same} % disable monospaced font for URLs
\usepackage[margin=1in]{geometry}
\usepackage{color}
\usepackage{fancyvrb}
\newcommand{\VerbBar}{|}
\newcommand{\VERB}{\Verb[commandchars=\\\{\}]}
\DefineVerbatimEnvironment{Highlighting}{Verbatim}{commandchars=\\\{\}}
% Add ',fontsize=\small' for more characters per line
\usepackage{framed}
\definecolor{shadecolor}{RGB}{248,248,248}
\newenvironment{Shaded}{\begin{snugshade}}{\end{snugshade}}
\newcommand{\AlertTok}[1]{\textcolor[rgb]{0.94,0.16,0.16}{#1}}
\newcommand{\AnnotationTok}[1]{\textcolor[rgb]{0.56,0.35,0.01}{\textbf{\textit{#1}}}}
\newcommand{\AttributeTok}[1]{\textcolor[rgb]{0.77,0.63,0.00}{#1}}
\newcommand{\BaseNTok}[1]{\textcolor[rgb]{0.00,0.00,0.81}{#1}}
\newcommand{\BuiltInTok}[1]{#1}
\newcommand{\CharTok}[1]{\textcolor[rgb]{0.31,0.60,0.02}{#1}}
\newcommand{\CommentTok}[1]{\textcolor[rgb]{0.56,0.35,0.01}{\textit{#1}}}
\newcommand{\CommentVarTok}[1]{\textcolor[rgb]{0.56,0.35,0.01}{\textbf{\textit{#1}}}}
\newcommand{\ConstantTok}[1]{\textcolor[rgb]{0.00,0.00,0.00}{#1}}
\newcommand{\ControlFlowTok}[1]{\textcolor[rgb]{0.13,0.29,0.53}{\textbf{#1}}}
\newcommand{\DataTypeTok}[1]{\textcolor[rgb]{0.13,0.29,0.53}{#1}}
\newcommand{\DecValTok}[1]{\textcolor[rgb]{0.00,0.00,0.81}{#1}}
\newcommand{\DocumentationTok}[1]{\textcolor[rgb]{0.56,0.35,0.01}{\textbf{\textit{#1}}}}
\newcommand{\ErrorTok}[1]{\textcolor[rgb]{0.64,0.00,0.00}{\textbf{#1}}}
\newcommand{\ExtensionTok}[1]{#1}
\newcommand{\FloatTok}[1]{\textcolor[rgb]{0.00,0.00,0.81}{#1}}
\newcommand{\FunctionTok}[1]{\textcolor[rgb]{0.00,0.00,0.00}{#1}}
\newcommand{\ImportTok}[1]{#1}
\newcommand{\InformationTok}[1]{\textcolor[rgb]{0.56,0.35,0.01}{\textbf{\textit{#1}}}}
\newcommand{\KeywordTok}[1]{\textcolor[rgb]{0.13,0.29,0.53}{\textbf{#1}}}
\newcommand{\NormalTok}[1]{#1}
\newcommand{\OperatorTok}[1]{\textcolor[rgb]{0.81,0.36,0.00}{\textbf{#1}}}
\newcommand{\OtherTok}[1]{\textcolor[rgb]{0.56,0.35,0.01}{#1}}
\newcommand{\PreprocessorTok}[1]{\textcolor[rgb]{0.56,0.35,0.01}{\textit{#1}}}
\newcommand{\RegionMarkerTok}[1]{#1}
\newcommand{\SpecialCharTok}[1]{\textcolor[rgb]{0.00,0.00,0.00}{#1}}
\newcommand{\SpecialStringTok}[1]{\textcolor[rgb]{0.31,0.60,0.02}{#1}}
\newcommand{\StringTok}[1]{\textcolor[rgb]{0.31,0.60,0.02}{#1}}
\newcommand{\VariableTok}[1]{\textcolor[rgb]{0.00,0.00,0.00}{#1}}
\newcommand{\VerbatimStringTok}[1]{\textcolor[rgb]{0.31,0.60,0.02}{#1}}
\newcommand{\WarningTok}[1]{\textcolor[rgb]{0.56,0.35,0.01}{\textbf{\textit{#1}}}}
\usepackage{graphicx}
\makeatletter
\def\maxwidth{\ifdim\Gin@nat@width>\linewidth\linewidth\else\Gin@nat@width\fi}
\def\maxheight{\ifdim\Gin@nat@height>\textheight\textheight\else\Gin@nat@height\fi}
\makeatother
% Scale images if necessary, so that they will not overflow the page
% margins by default, and it is still possible to overwrite the defaults
% using explicit options in \includegraphics[width, height, ...]{}
\setkeys{Gin}{width=\maxwidth,height=\maxheight,keepaspectratio}
% Set default figure placement to htbp
\makeatletter
\def\fps@figure{htbp}
\makeatother
\setlength{\emergencystretch}{3em} % prevent overfull lines
\providecommand{\tightlist}{%
  \setlength{\itemsep}{0pt}\setlength{\parskip}{0pt}}
\setcounter{secnumdepth}{-\maxdimen} % remove section numbering
\ifluatex
  \usepackage{selnolig}  % disable illegal ligatures
\fi

\title{STAT 608 - Assignment1}
\author{Jack Rodoni}
\date{1/28/2022}

\begin{document}
\maketitle

\hypertarget{question-1-page-38-in-our-textbook.-notice-that-to-answer-these-questions-correctly-you-should-be-thinking-like-a-statistician-and-talking-about-population-parameters-not-only-sample-statistics.-that-is-do-some-inference-in-every-part-using-as-much-everyday-laypersons-terminology-as-possible.-for-example-part-b-should-not-just-say-the-intercept-is-or-is-not-10000.-what-does-an-intercept-mean-in-context-to-someone-who-is-selling-movie-tickets-use-the-discussion-board-as-needed-to-get-all-your-details-correct.}{%
\section{1. Question 1, page 38 in our textbook. Notice that to answer
these questions correctly, you should be thinking like a statistician
and talking about population parameters, not only sample statistics.
That is, do some inference in every part, using as much everyday
layperson's terminology as possible. For example, part b should not just
say ``The intercept is (or is not) 10,000.'' What does an intercept mean
in context to someone who is selling movie tickets? Use the discussion
board as needed to get all your details
correct.}\label{question-1-page-38-in-our-textbook.-notice-that-to-answer-these-questions-correctly-you-should-be-thinking-like-a-statistician-and-talking-about-population-parameters-not-only-sample-statistics.-that-is-do-some-inference-in-every-part-using-as-much-everyday-laypersons-terminology-as-possible.-for-example-part-b-should-not-just-say-the-intercept-is-or-is-not-10000.-what-does-an-intercept-mean-in-context-to-someone-who-is-selling-movie-tickets-use-the-discussion-board-as-needed-to-get-all-your-details-correct.}}

\begin{enumerate}
\def\labelenumi{\arabic{enumi}.}
\tightlist
\item
  The web site www.playbill.com provides weekly reports on the box
  office ticket sales for plays on Broadway in New York. We shall
  consider the data for the week October 11--17, 2004 (referred to below
  as the current week). The data are in the form of the gross box office
  results for the current week and the gross box office results for the
  previous week (i.e., October 3--10, 2004). The data, plotted in Figure
  2.6 , are available on the book web site in the file playbill.csv.
\end{enumerate}

Fit the following model to the data: Y= b\textsubscript{0}
+b\textsubscript{1}X where Y is the gross box office results for the
current week (in \$) and x is the gross box office results for the
previous week (in \$). Complete the following tasks:

\begin{Shaded}
\begin{Highlighting}[]
\NormalTok{playbill }\OtherTok{\textless{}{-}} \FunctionTok{read.csv}\NormalTok{(}\FunctionTok{paste0}\NormalTok{(}\StringTok{"C:/Users/jackr/OneDrive/Desktop/Graduate School "}\NormalTok{,}
                     \StringTok{"Courses/STAT 608 {-} Regression Analysis/Data/playbill.csv"}\NormalTok{))}

\NormalTok{lm1 }\OtherTok{=} \FunctionTok{with}\NormalTok{(playbill,}\FunctionTok{lm}\NormalTok{(CurrentWeek}\SpecialCharTok{\textasciitilde{}}\NormalTok{LastWeek))}

\NormalTok{modelsum }\OtherTok{=} \FunctionTok{summary}\NormalTok{(lm1)}
\FunctionTok{summary}\NormalTok{(lm1)}
\end{Highlighting}
\end{Shaded}

\begin{verbatim}
## 
## Call:
## lm(formula = CurrentWeek ~ LastWeek)
## 
## Residuals:
##    Min     1Q Median     3Q    Max 
## -36926  -7525  -2581   7782  35443 
## 
## Coefficients:
##              Estimate Std. Error t value Pr(>|t|)    
## (Intercept) 6.805e+03  9.929e+03   0.685    0.503    
## LastWeek    9.821e-01  1.443e-02  68.071   <2e-16 ***
## ---
## Signif. codes:  0 '***' 0.001 '**' 0.01 '*' 0.05 '.' 0.1 ' ' 1
## 
## Residual standard error: 18010 on 16 degrees of freedom
## Multiple R-squared:  0.9966, Adjusted R-squared:  0.9963 
## F-statistic:  4634 on 1 and 16 DF,  p-value: < 2.2e-16
\end{verbatim}

\begin{Shaded}
\begin{Highlighting}[]
\NormalTok{equatiomatic}\SpecialCharTok{::}\FunctionTok{extract\_eq}\NormalTok{(lm1, }\AttributeTok{use\_coefs =} \ConstantTok{TRUE}\NormalTok{)}
\end{Highlighting}
\end{Shaded}

\begin{equation}
\operatorname{\widehat{CurrentWeek}} = 6804.89 + 0.98(\operatorname{LastWeek})
\end{equation}

\begin{enumerate}
\def\labelenumi{(\alph{enumi})}
\tightlist
\item
  Find a 95\% confidence interval for the slope of the regression model,
  b\textsubscript{1}. Is 1 a plausible value for b\textsubscript{1}?
  Give a reason to support your answer.
\end{enumerate}

\begin{Shaded}
\begin{Highlighting}[]
\CommentTok{\# calculation done by R}
\FunctionTok{confint}\NormalTok{(lm1, }\AttributeTok{level =} \FloatTok{0.95}\NormalTok{)}

\CommentTok{\# if we just want B1}
\FunctionTok{confint}\NormalTok{(lm1, }\StringTok{"playbill$LastWeek"}\NormalTok{, }\AttributeTok{level =} \FloatTok{0.95}\NormalTok{)}
\end{Highlighting}
\end{Shaded}

\begin{Shaded}
\begin{Highlighting}[]
\CommentTok{\# By Hand calculation}
\FunctionTok{c}\NormalTok{(lm1}\SpecialCharTok{$}\NormalTok{coefficients[}\DecValTok{2}\NormalTok{] }\SpecialCharTok{{-}} \FunctionTok{qt}\NormalTok{(.}\DecValTok{975}\NormalTok{,}\DecValTok{16}\NormalTok{)}\SpecialCharTok{*}\NormalTok{modelsum}\SpecialCharTok{$}\NormalTok{coefficients[}\DecValTok{2}\NormalTok{,}\DecValTok{2}\NormalTok{],}
\NormalTok{  lm1}\SpecialCharTok{$}\NormalTok{coefficients[}\DecValTok{2}\NormalTok{] }\SpecialCharTok{+} \FunctionTok{qt}\NormalTok{(.}\DecValTok{975}\NormalTok{,}\DecValTok{16}\NormalTok{)}\SpecialCharTok{*}\NormalTok{modelsum}\SpecialCharTok{$}\NormalTok{coefficients[}\DecValTok{2}\NormalTok{,}\DecValTok{2}\NormalTok{])}
\end{Highlighting}
\end{Shaded}

\begin{verbatim}
##                     2.5 %       97.5 %
## (Intercept) -1.424433e+04 27854.099443
## LastWeek     9.514971e-01     1.012666
\end{verbatim}

\textbf{Yes, 1 is a reasonable value for B\textsubscript{1} as 1 is
contained in our confidence interval}

\begin{enumerate}
\def\labelenumi{(\alph{enumi})}
\setcounter{enumi}{1}
\tightlist
\item
  Test the null hypothesis H\textsubscript{0} : b\textsubscript{0} =
  10000 against a two-sided alternative. Interpret your result.
\end{enumerate}

\begin{Shaded}
\begin{Highlighting}[]
\NormalTok{teststat }\OtherTok{=}\NormalTok{ (lm1}\SpecialCharTok{$}\NormalTok{coefficients[}\DecValTok{1}\NormalTok{] }\SpecialCharTok{{-}} \DecValTok{10000}\NormalTok{)}\SpecialCharTok{/}\NormalTok{modelsum}\SpecialCharTok{$}\NormalTok{coefficients[}\DecValTok{1}\NormalTok{,}\DecValTok{2}\NormalTok{]}
\DecValTok{2}\SpecialCharTok{*}\NormalTok{(}\DecValTok{1}\SpecialCharTok{{-}}\NormalTok{(}\FunctionTok{pt}\NormalTok{(}\FunctionTok{abs}\NormalTok{(teststat),}\DecValTok{16}\NormalTok{))) }\CommentTok{\# See STAT 641 H.O 12 pg 24}
\end{Highlighting}
\end{Shaded}

\begin{verbatim}
## (Intercept) 
##   0.7517807
\end{verbatim}

\textbf{The p-value is 0.7517807 -\textgreater{} we would fail to reject
the null that b\textsubscript{0} = 10000}

\begin{enumerate}
\def\labelenumi{(\alph{enumi})}
\setcounter{enumi}{2}
\tightlist
\item
  Use the fitted regression model to estimate the gross box office
  results for the current week (in \$) for a production with \$400,000
  in gross box office the previous week. Find a 95\% prediction interval
  for the gross box office. is \$450,000 a feasible value for the gross
  box office results in the current week, for a production with
  \$400,000 in gross box office the previous week? Give a reason to
  support your answer.
\end{enumerate}

\begin{Shaded}
\begin{Highlighting}[]
\CommentTok{\# 95\% CI}
\FunctionTok{summary}\NormalTok{(lm1)}
\DocumentationTok{\#\# R calculation}
\FunctionTok{predict}\NormalTok{(lm1, }\AttributeTok{newdata =} \FunctionTok{data.frame}\NormalTok{(}\AttributeTok{LastWeek =} \DecValTok{400000}\NormalTok{), }\AttributeTok{interval =} \StringTok{"confidence"}\NormalTok{)}
\DocumentationTok{\#\# by hand}
\FunctionTok{with}\NormalTok{(playbill, }\DecValTok{399637} \SpecialCharTok{+} \FunctionTok{c}\NormalTok{(}\SpecialCharTok{{-}}\DecValTok{1}\NormalTok{, }\DecValTok{1}\NormalTok{) }\SpecialCharTok{*} \FunctionTok{qt}\NormalTok{(}\FloatTok{0.975}\NormalTok{, }\DecValTok{16}\NormalTok{) }\SpecialCharTok{*} \DecValTok{18010} \SpecialCharTok{*} 
  \FunctionTok{sqrt}\NormalTok{((}\DecValTok{1}\SpecialCharTok{/}\DecValTok{18}\NormalTok{) }\SpecialCharTok{+}\NormalTok{ (}\DecValTok{400000} \SpecialCharTok{{-}} \FunctionTok{mean}\NormalTok{(LastWeek)) }\SpecialCharTok{\^{}} \DecValTok{2} \SpecialCharTok{/} \FunctionTok{sum}\NormalTok{((LastWeek }\SpecialCharTok{{-}} \FunctionTok{mean}\NormalTok{(LastWeek)) }\SpecialCharTok{\^{}} \DecValTok{2}\NormalTok{)))}


\CommentTok{\# 95\% PI}
\DocumentationTok{\#\# R calculation}
\FunctionTok{predict}\NormalTok{(lm1, }\AttributeTok{newdata =} \FunctionTok{data.frame}\NormalTok{(}\AttributeTok{LastWeek =} \DecValTok{400000}\NormalTok{), }\AttributeTok{interval =} \StringTok{"prediction"}\NormalTok{)}
\DocumentationTok{\#\# by hand}
\NormalTok{PI }\OtherTok{=} \FunctionTok{with}\NormalTok{(playbill, }\DecValTok{399637} \SpecialCharTok{+} \FunctionTok{c}\NormalTok{(}\SpecialCharTok{{-}}\DecValTok{1}\NormalTok{, }\DecValTok{1}\NormalTok{) }\SpecialCharTok{*} \FunctionTok{qt}\NormalTok{(}\FloatTok{0.975}\NormalTok{, }\DecValTok{16}\NormalTok{) }\SpecialCharTok{*} \DecValTok{18010} \SpecialCharTok{*} 
  \FunctionTok{sqrt}\NormalTok{(}\DecValTok{1}\SpecialCharTok{+}\NormalTok{(}\DecValTok{1}\SpecialCharTok{/}\DecValTok{18}\NormalTok{) }\SpecialCharTok{+}\NormalTok{ (}\DecValTok{400000} \SpecialCharTok{{-}} \FunctionTok{mean}\NormalTok{(LastWeek)) }\SpecialCharTok{\^{}} \DecValTok{2} \SpecialCharTok{/} \FunctionTok{sum}\NormalTok{((LastWeek }\SpecialCharTok{{-}} \FunctionTok{mean}\NormalTok{(LastWeek)) }\SpecialCharTok{\^{}} \DecValTok{2}\NormalTok{)))}
\end{Highlighting}
\end{Shaded}

\textbf{450,000 is not a feasible value for the gross box office results
in the current week for a production with \$400,000 in gross box office
the previous week}

\begin{verbatim}
## [1] 359826.9 439447.1
\end{verbatim}

\begin{enumerate}
\def\labelenumi{(\alph{enumi})}
\setcounter{enumi}{3}
\tightlist
\item
  Some promoters of Broadway plays use the prediction rule that next
  week's gross box office results will be equal to this week's gross box
  office results. Comment on the appropriateness of this rule.
\end{enumerate}

\textbf{This rule seems reasonable as our last weeks value is included
in our 95\% prediction interval for next week}

\hypertarget{show-that-varyixi-xi-vari-in-the-simple-linear-regression-model.-dont-overthink-this-the-answer-is-simple.-what-did-you-assume-in-answering-this}{%
\section{\texorpdfstring{2. Show that
Var(Y\textsubscript{i}\textbar X\textsubscript{i} = x\textsubscript{i})
= Var(\$\epsilon\textsubscript{i}) in the simple linear regression
model. Don't overthink this; the answer is simple. What did you assume
in answering
this?}{2. Show that Var(Yi\textbar Xi = xi) = Var(\$i) in the simple linear regression model. Don't overthink this; the answer is simple. What did you assume in answering this?}}\label{show-that-varyixi-xi-vari-in-the-simple-linear-regression-model.-dont-overthink-this-the-answer-is-simple.-what-did-you-assume-in-answering-this}}

\newpage

\hypertarget{define-using-only-words-what-the-least-squares-criterion-is.}{%
\section{3. Define using only words what the least squares criterion
is.}\label{define-using-only-words-what-the-least-squares-criterion-is.}}

Simply put, the least squares criterion is when we are fitting a
regression model we select the model that minimizes the sum of the
square errors. In other words, we minimize the sum of the squares of the
projections from our data to the model space.

\hypertarget{question-4-page-40-in-our-textbook-except-do}{%
\section{4. Question 4, page 40 in our textbook, except
do:}\label{question-4-page-40-in-our-textbook-except-do}}

\begin{enumerate}
\def\labelenumi{\arabic{enumi}.}
\setcounter{enumi}{3}
\tightlist
\item
  Straight-line regression through the origin: In this question we shall
  make the following assumptions:
\end{enumerate}

\begin{enumerate}
\def\labelenumi{(\arabic{enumi})}
\tightlist
\item
  Y is related to x by the simple linear regression model:
  Y\textsubscript{i} = BX\textsubscript{i}+e\textsubscript{i} (i =
  1,2,3,\ldots) i.e.~E(Y\textbar X=x\textsubscript{i}) =
  BX\textsubscript{i}
\item
  The errors
  e\textsubscript{1},e\textsubscript{2},\ldots,e\textsubscript{n} are
  independent of each other
\item
  The errors
  e\textsubscript{1},e\textsubscript{2},\ldots,e\textsubscript{n} have a
  common variance sigma\textsuperscript{2}
\item
  The errors are normally distributed with a mean of 0 and variance
  sigma\textsuperscript{2} (especially when the sample size is small),
  i.e., (e\textbar X)\textasciitilde{}\N(0, sigma\textsuperscript{2})
\end{enumerate}

\begin{enumerate}
\def\labelenumi{(\alph{enumi})}
\tightlist
\item
  Setup:
\end{enumerate}

\begin{enumerate}
\def\labelenumi{\roman{enumi}.}
\tightlist
\item
  Write down your design matrix X.\\
  ~\\
  ~\\
  ~\\
  ~\\
  ~\\
\item
  Show, using matrix notation and starting with the principle of least
  squares, that the least squares estimate of B is given by:\\
  ~\\
  ~\\
  ~\\
  ~\\
  ~\\
  ~\\
  ~\\
  ~\\
  ~\\
  ~\\
  ~\\
  ~\\
  ~\\
  ~\\
  ~\\
  ~\\
  ~\\
  ~\\
  ~\\
  ~\\
  ~\\
  ~\\
  ~\\
  ~\\
\end{enumerate}

\begin{enumerate}
\def\labelenumi{(\alph{enumi})}
\setcounter{enumi}{1}
\tightlist
\item
  Under the above assumptions show that:
\end{enumerate}

\begin{enumerate}
\def\labelenumi{\roman{enumi}.}
\item
  \hfill\break
  \hfill\break
  \hfill\break
  \hfill\break
  \hfill\break
  \hfill\break
  \hfill\break
  \hfill\break
  \hfill\break
  \hfill\break
  \hfill\break
  \hfill\break
  \hfill\break
  \hfill\break
  \hfill\break
  \hfill\break
  \hfill\break
  \hfill\break
  \hfill\break
  \hfill\break
\item
  \hfill\break
  \hfill\break
  \hfill\break
  \hfill\break
  \hfill\break
  \hfill\break
  \hfill\break
  \hfill\break
  \hfill\break
  \hfill\break
  \hfill\break
  \hfill\break
  \hfill\break
  \hfill\break
  \hfill\break
  \hfill\break
  \hfill\break
  \hfill\break
\end{enumerate}

\end{document}
